%\documentclass[show notes]{beamer}
%\documentclass[handout]{beamer}
\documentclass[]{beamer}

\usepackage{pgfpages}
\usepackage[utf8]{inputenc}
\usepackage[T1]{fontenc}
%\usepackage{mathabx}
%\usepackage{mathpazo}
%\usepackage{eulervm}
%\usepackage{natbib}
\usepackage{adjustbox}
\usepackage{booktabs}
%\usepackage{svg}
\usepackage{colortbl}
\usepackage{hyperref}
\hypersetup{colorlinks=true, urlcolor=uog, linkcolor=uog, citecolor=uog}

\usepackage[backend=biber, style=authoryear, maxbibnames=99, dashed=false]{biblatex}
\addbibresource{tagged-SWCD2021.bib}


\usepackage{caption}
\captionsetup[figure]{labelformat=empty}% redefines the caption setup of the figures environment in the beamer class.


\usetheme{Madrid}
\definecolor{uog}{rgb}{0,.5,0}
\usecolortheme[named=uog]{structure}

\mode<handout>{
	\pgfpagesuselayout{4 on 1}[letterpaper] 
	\setbeameroption{show notes}
}


% The following code uses \AtBeginSection to place a frame with the section title (\insertsectionhead) inside a beamercolorbox.
% From https://tex.stackexchange.com/questions/178800/creating-sections-each-with-title-pages-in-beamers-slides
\AtBeginSection[]{
	\begin{frame}
		\vfill
		\centering
		\begin{beamercolorbox}[sep=8pt,center,shadow=true,rounded=true]{title}
			\usebeamerfont{title}\insertsectionhead\par%
		\end{beamercolorbox}
		\vfill
	\end{frame}
}

\title[Invasive Species Issues on Guam]{Overview of Invasive Species Issues on Guam}

\author[]{
	Glenn Dulla\textsuperscript{1}, Roland Quitugua\textsuperscript{2} and Aubrey Moore\textsuperscript{2}\\
	\bigskip 
	\textsuperscript{1}Guam Department of Agriculture,
    \textsuperscript{2}University of Guam
}

%\institute[]{College of Natural and Applied Sciences\\University of Guam}

%\titlegraphic{\includegraphics[width=2cm]{big_g2.pdf}}

\date[]{Pacific Ecological Security Conference\\Palau, October 6, 2022\\ \tiny\url{https://github.com/aubreymoore/SWDC-2021-07-30/raw/main/SWCD-2021-07-30.pdf}}










%\documentclass{beamer}
%\usetheme{Madrid}
%
%\title{Overview of Invasive Species Issues on Guam}
%\author{Glenn Dulla, Roland Quitugua, Aubrey Moore}
%\date{Prepared for the ...}
\begin{document}
	
%\begin{frame}
    \maketitle
%\end{frame}

\begin{frame}{Hafa Adai}
	\adjincludegraphics[height=1.05\textheight,center]{graphics/Guam.jpg}
\end{frame}

\begin{frame}{Priority Issues}
	Guam's natural ecosystems, especially Guam's forests, are rapidly being destroyed by invasive species.
	
	
\end{frame}

%\subsection{Priority Issue 1: Brown treesnake}

\begin{frame}{Priority Issue 1: Brown treesnake}
\end{frame}

%\subsection{Cycad aulacaspis scale CAS}

\begin{frame}{Priority Issue 2: Cycad scale insect (CAS)}
\end{frame}


\begin{frame}{Priority Issue 3: Coconut rhinoceros beetle (CRB)}
\end{frame}

\begin{frame}{Priority Issue 4: Little fire ant (LFA)}
\end{frame}

\begin{frame}{Challenges}
	\begin{itemize}
		\item Professional capacity is low.
		\item Guam suffers from the \textit{taxonomic impediment}
		\item Guam does not have a terrestrial biodiversity inventory
	\end{itemize}
\end{frame}

\begin{frame}{Funding sources}
	\begin{itemize}
	   \item Department of Interior - Office of Insular Affairs
	   \item USDA - Forest Service
	   \item USDA - APHIS
	   \item DOD
	   \item Government of Guam - Invasive Species Tarrif
	\end{itemize}		
\end{frame}

\begin{frame}{National/Territorial Invasive Species Plans}
	Regional Biosecurity Plan for Micronesia and Hawaii
   \url{https://pacific.navfac.navy.mil/About-Us/Regional-Biosecurity-Plan-for-Micronesia-and-Hawaii/}
\end{frame}
\begin{frame}{Next steps}
	Coming soon.
\end{frame}

\end{document}
